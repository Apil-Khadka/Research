\section{Implementation}\label{sec:implementation}

\subsection*{Overview of the Digital Home System}
The digital home system is designed to automate and control household switches using relays, ensuring a secure and fast wireless communication protocol that supports control via cards and mobile phones. The implementation leverages readily available electronic components and follows a modular approach to ensure scalability and reliability.

\subsection*{Hardware Components}
The primary hardware components for the system include:

\begin{itemize}
	\item \textbf{Relays:} Solid-State Relays (SSRs) are used for silent and reliable switching of household appliances.
	\item \textbf{Microcontroller:} An ESP32 microcontroller is chosen for its support for Wi-Fi and Bluetooth Low Energy (BLE).
	\item \textbf{Wireless Modules:} NFC (Near Field Communication) modules for card-based control and Bluetooth for mobile phone integration.
	\item \textbf{Power Supply Unit:} A regulated DC power supply for the microcontroller and relays.
	\item \textbf{Protective Components:} Circuit breakers and fuses to ensure safety in the event of a fault.
	\item \textbf{Sensors (Optional):} Environmental sensors (e.g., temperature, humidity) to extend functionality.
\end{itemize}

\subsection*{Software Design}
The software system is designed to facilitate seamless communication between the user and the digital home system. Key components include:

\begin{itemize}
	\item \textbf{Microcontroller Firmware:} The firmware, written in C/C++, controls the relays based on user commands received via wireless communication.
	\item \textbf{Mobile Application:} A mobile app developed in Kotlin or React Native provides a user-friendly interface for controlling devices.
	\item \textbf{Authentication Mechanisms:} Secure protocols such as WPA3 for Wi-Fi communication and AES encryption for NFC/Bluetooth.
	\item \textbf{Cloud Integration (Optional):} Integration with cloud platforms like Firebase for remote control and monitoring.
\end{itemize}

\subsection*{Wireless Communication}
The system uses the following wireless technologies:

\begin{enumerate}
	\item \textbf{Wi-Fi:} For long-range control and integration with internet-based services.
	\item \textbf{Bluetooth Low Energy (BLE):} For short-range, low-power control via mobile phones.
	\item \textbf{NFC/RFID:} For card-based access to control switches securely and quickly.
\end{enumerate}

\subsection*{System Integration}
The implementation process follows these steps:

\begin{enumerate}
	\item \textbf{Hardware Assembly:} Connect relays to the ESP32 microcontroller, ensuring proper isolation and safety.
	\item \textbf{Firmware Development:} Program the ESP32 to control relays based on inputs received via Wi-Fi, BLE, or NFC.
	\item \textbf{Mobile App Development:} Develop an intuitive app that allows users to configure and control devices, authenticate access, and monitor system status.
	\item \textbf{Security Measures:} Implement robust encryption and authentication protocols to prevent unauthorized access.
	\item \textbf{Testing:} Conduct rigorous testing of the system under various conditions to ensure reliability and responsiveness.
\end{enumerate}

\subsection*{Engineering for Nepal}
The system is tailored to meet the specific needs and challenges of the Nepalese market:

\begin{itemize}
	\item \textbf{Voltage Compatibility:} Designed for Nepal's standard 220V AC household voltage.
	\item \textbf{Localized Manufacturing:} Utilize locally available components to reduce costs and support the local economy.
	\item \textbf{Affordable Design:} Optimize the system for cost-effectiveness without compromising on functionality or safety.
	\item \textbf{Urban and Rural Deployment:} Ensure adaptability to both urban homes with high-speed internet and rural areas with limited connectivity.
\end{itemize}

\subsection*{Future Enhancements}
To enhance the system's capabilities, the following features can be explored:

\begin{itemize}
	\item \textbf{Voice Control:} Integrate voice assistants like Google Assistant or Alexa for hands-free control.
	\item \textbf{Energy Monitoring:} Add power consumption sensors to track and optimize energy usage.
	\item \textbf{Integration with Smart Devices:} Extend compatibility with other smart home devices such as smart thermostats and cameras.
	\item \textbf{Cloud-Based Analytics:} Use data analytics to provide insights and predictive maintenance.
\end{itemize}
